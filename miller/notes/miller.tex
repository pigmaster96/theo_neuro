\documentclass{report}
\title{An Introductory Course in Computational Neuroscience---Paul Miller (Notes)}
\date{Started 14 Dec 2024}
\author{Malcolm}
\usepackage{amsmath} %import math
\usepackage{mathtools} %more math
\usepackage{amssymb} %for QED symbol
\usepackage{amsthm} %
\usepackage{bm} %bolding without changing font
\usepackage{graphicx} %import imaging
\graphicspath{{./images/}} %set imaging path
\newcommand*{\vertbar}{\rule[-1ex]{0.5pt}{2.5ex}} %matrix
\newcommand*{\horzbar}{\rule[.5ex]{2.5ex}{0.5pt}} %matrix
\begin{document}
\maketitle
\tableofcontents
\newpage

\section{LIF}
\subsection{Formula}
The \textit{Nernst potential} $E_A$ of an ion $A$ of charge $z_A$ with intracellular concentration $[A_{\text{in}}]$ and extracellular concentration $[A_{\text{out}}]$ is given by
\begin{equation*}
E_A=\frac{k_BT}{z_Aq_e}\ln\left(\frac{[A_{\text{out}}]}{[A_{\text{in}}]}\right)
\end{equation*}
where $T$ is the temperature in Kelvin, $k_B$ the Boltzmann constant $(1.39\times10^{-23}JK^{-1})$ (which converts units of temperature to units of thermal energy). $q_e$ is the fundamental electronic charge
$(1.60\times10^{-19}C)$.
The current through a channel is given by
\begin{equation*}
I_t=G_t(V_m-E_t)
\end{equation*}
Where $G_t$ represents conductance and $E_t$ the nernst potential; $t$ represents the type of channel. The total membrane current $I_m$ can be modelled as











\end{document}
